\section{Wnioski}
Pracę ze środowiskiem MATLAB Simulink uznajemy za niebywale wydajną oraz skuteczną w tworzeniu prototypów układów regulacji. Ze względu na wprowadzenie uzasadnionych uproszczeń w modelu matematycznym wahadła proces identyfikacji systemu przebiegł sprawnie i stosunkowo szybko. Kluczowym elementem było pominięcie wpływu oddziaływanie wahadła na wózek w niestabilnym położeniu równowagi, dzięki czemu mogliśmy zidentyfikować dynamikę wózka zupełnie niezależnie. Uniknęliśmy konieczności czasochłonnych pomiarów rezystancji silnika oraz stałych - elektrycznej i mechanicznej. Uzyskany model zadowalająco dobrze odzwierciedlał dynamikę rzeczywistego obiektu czego dowód przedstawiono na rysunkach \ref{fig:wozek}, \ref{fig:poW} i \ref{fig:prW}. Największe problemy sprawiło nam tarcie statyczne, którego efektów nie udało nam się wyeliminować pomimo różnorodności obranych prób. System ze względu na mocne nieliniowości nie jest prostym obiektem do sterowania. Warto podkreślić, że regulator LQR pomimo tego znakomicie poradził sobie z tym zadaniem. Przy syntezie regulatora kluczowym okazał się właściwy dobór macierzy Q oraz R wchodzących w skład wskaźnika jakości. Ważne, aby wartości jakie przyjmuje sterowanie mieściły się w dopuszczalnych granicach, w przypadku wahadła od -0.5 do 0.5. Ze względu na ograniczoną długość szyny, po której może poruszać się wózek istotne staje się ustalenie amplitudy sterowania wzbudzającego system w algorytmie swing-up. Zbyt duże może spowodować wyjechanie wózka poza dopuszczalne granice, natomiast zbyt małe znacznie wydłuża czas działania.