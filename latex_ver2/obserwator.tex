\section{Obserwator stanu}
\subsection{Obserwator Luenbergera}
Po szczegółowej analizie zarejestrowanych przebiegów zauważono duży wpływ szumu kwantyzacji na jakość regulacji, szczególnie w przypadku prędkości kątowej wahadła wyliczanej przez iloraz różnicowy położenia wahadła w kolejnych chwilach czasu. W celu wygładzenia przebiegu prędkości zaprojektowano obserwator Luenbergera dla systemu zlinearyzowanego w taki sposób, aby błąd estymacji prędkości dostatecznie szybko zdążał do zera. Doświadczalnie wyznaczono odpowiednie ujemne wartości własne macierzy:
\begin{equation}  
eig(A-LC) = \left\lbrace wartosci dopisac \right\rbrace 
\end{equation} 
Zdecydowano się na obserwator pełnego rzędu, natomiast do regulatora podano jedynie estymacje stanów obarczonych największym błędem pomiarowym - prędkości wahadła i wózka.
% przebiegi porównawcze

\subsection{Model ARMAX}
